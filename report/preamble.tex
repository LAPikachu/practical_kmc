\documentclass{scrartcl} % KOMA Dokumentenklasse mit DIN-Papier als Standard
\usepackage[a4paper]{geometry} % Paket zur flexiblen Anpassung des Seitenlayouts
\savegeometry{default}
\usepackage[autooneside=false]{scrlayer-scrpage} % Paket zur flexiblen Anpassung der Kopf- und Fußzeilen
%
%\usepackage[tocindentauto]{tocstyle} % Paket zur flexiblen Anpassung der Formatierung des Inhaltsverzeichnisses (wird hier nur benötigt, weil einmalig die Nummerierung auf römische Zahlen umgestellt wurde)
 %\usetocstyle{KOMAlike}
%
\usepackage[english]{babel} % Sprachenpaket zur Anpassung der Standardausgaben von bestimmten Schlüsselwörtern und Aktivierung der deutschen Silbentrennung
%
\usepackage{miller}%to intuitively display miller indices%

\usepackage{parskip} %gets rid of hbox warning - just add empty lines to imply linebreaks

\usepackage[utf8]{inputenc} % Kodierungspaket zur Festlegung, wie die eingegebenen Zeichen (Code hier im Dokument) interpretiert werden sollen
%
\usepackage{csquotes}

\usepackage[T1]{fontenc} % Kodierungspaket zur Festlegung, wie die ausgegebenen Zeichen (Buchstaben im PDF-Dokument) implementiert werden sollen
%
\usepackage{lipsum} % Paket zur Einbindung von lateinischen Blindtexten
%
\usepackage{multicol} % Paket zur flexiblen Erstellung mehrspaltiger Texte und zum Zusammenfassen von Spalten in Tabellen
%
\usepackage{ragged2e} % Paket zur Verbesserung der Darstellung von linksbündigem, rechtsbündigem und zentriertem Text (erlaubt Silbentrennung)
%
\usepackage{quoting} % Paket zur ansprechenden Darstellung von Zitaten
%
\usepackage[marginal,norule]{footmisc} % Paket zur flexiblen Anpassung der Formatierung von Fußnoten
%
\usepackage{listings} % Paket zur vereinfachten Einbindung und zur ansprechenden Darstellung von Code in beliebigen Programmiersprachen
%
\usepackage[section, below]{placeins}
%\usepackage[section]{placeins}
%
%
\usepackage{microtype} % Paket zur Verbesserung der Mikro-Typographie durch Anpassung von Zeilenenden und sinnvollen Reduzierung von Silbentrennungen
\usepackage{lmodern} % Schriftenpaket für Latin Modern Schriften (beliebig skalierbare, richtig implementierte Vektorschriften)
%\renewcommand{\rmdefault}{lmss} %to use sans serif by default
\usepackage{textcomp} % Paket zur Erweiterung von verfügbaren Sonderzeichen (Copyright, Trademark, ...)
%
\usepackage{textgreek} %to use greek letters in text
%
\usepackage{amsmath} % Extrem umfassendes Mathematik-Paket der American Math Society
% allow for wider matrices
\newcommand{\tens}[1]{\boldsymbol{\mathsf{#1}}}
\setcounter{MaxMatrixCols}{20}
\usepackage{mathtools} % Erweiterung des amsmath-Pakets (enthält amsmath-Paket)
% Ergänzungen für Mathe:
\DeclarePairedDelimiter\abs{\lvert}{\rvert}%
\DeclarePairedDelimiter\norm{\lVert}{\rVert}%
%
%
% Anpassung von Klammern 
% Swap the definition of \abs* and \norm*, so that \abs
% and \norm resizes the size of the brackets, and the 
% starred version does not.
\makeatletter
\let\oldabs\abs
\def\abs{\@ifstar{\oldabs}{\oldabs*}}
%
\let\oldnorm\norm
\def\norm{\@ifstar{\oldnorm}{\oldnorm*}}
\makeatother
%
%
\usepackage{amsfonts} % Schriftenpaket der American Math Society zur Erweiterung der Schriften in Mathe-Umgebungen
%
\usepackage{amssymb} % Paket zur Erweiterung der verfügbaren Mathe-Symbole
%
\usepackage{MnSymbol} % Paket zur Erweiterung der verfügbaren Mathe-Symbole
%
\usepackage{graphicx} % Paket zur vereinfachten und flexiblen Einbindung von Grafiken (jpg, png, pdf)
%
\usepackage{pdfpages} % Paket zur vereinfachten und flexiblen Einbindung von mehrseitigen PDFs
%
\usepackage{chemformula} %Paket für chem. Formeln e.g. with \ce or \ch and stuff in brackets
%
\usepackage{xcolor} % Farbenpaket zur vereinfachten Einbindung und Anpassung von Farben
%
\usepackage{booktabs} % Paket zur Verbesserung der Darstellung von horizontalen Linien in Tabellen
%
\usepackage{multirow} % Paket zum Zusammenfassen von Zeilen in Tabellen
%
\usepackage{tikz} % Paket zur Erstellung von ansprechenden und anspruchsvollen Zeichnungen innerhalb von LaTeX
%\usepackage{pgfplots} % Paket zur Erstellung von ansprechenden und anspruchsvollen Plots innerhalb von LaTeX
%
\usepackage[style=nature, backend=biber]{biblatex} % Paket zur vereinfachten und automatisierten Erstellung eines Literaturverzeichnisses aus einer externen Datenbank

%
\usepackage{xparse} % Paket zur Erweiterung der Funktionalitäten bei der Definition von neuen Befehlen und Umgebungen
%
\usepackage{hyperref} % Paket zur Erstellung und Anpassung von anklickbaren, intelligenten Querverweisen innerhalb des Dokuments und Hyperlinks (Weblinks, Maillinks, Filelinks)
%
\def\code#1{\texttt{#1}} % to display cod in monospace using \code{arg1}
%
%
%
%
% Einstellungen für das Dokument:
%
 \KOMAoptions{headsepline, footsepline}
 \setkomafont{headsepline}{\color{gray}}
 \setkomafont{footsepline}{\color{gray}}
 \setkomafont{pagehead}{\scshape}
 \setkomafont{pagefoot}{} %\bfseries
 \automark[subsection]{section}
 \lohead{}
 \cohead{\rightmark}
 \rohead{}
 \lofoot{}
 \cofoot{\pagemark}
 \rofoot{}
%
%
\definecolor{meinSchwarz} {RGB} {0,0,0}
%
\hypersetup{linktoc = all, colorlinks ,  citecolor = meinSchwarz , linkcolor = blue, urlcolor  = meinSchwarz , filecolor = meinSchwarz }
%
%\addto\extrasngerman{\def\figureautorefname{Abb.}}
%
\setlength{\parindent}{0pt}
\setlength{\parskip}{0.5\baselineskip plus 0.2\baselineskip minus 0.1\baselineskip}
\linespread{1.25}
%
\newcommand{\degcel}{\,\textcelsius{} }
%
% Angaben zu Variablen
\graphicspath{{graphics}}
\bibliography{kmc_practical}
%